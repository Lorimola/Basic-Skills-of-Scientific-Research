\documentclass[a4paper,12pt]{article}
\usepackage{ctex}  % 使用中文支持
\usepackage{geometry}
\geometry{a4paper, left=2.5cm, right=2.5cm, top=2.5cm, bottom=2.5cm}
\usepackage{amsmath, amssymb}
\usepackage{enumerate}

\title{GUI-R1  A Generalist R1-Style Vision-Language Action Model For GUI Agents}
\author{Yang}

\begin{document}
\maketitle
\section{Introduction}
当下使用LVLM发展GUI Agent,特点是只依赖屏幕分析作为信息源进行决策,不依赖环境的文本描述,在决策上具有更高的灵活性。屏幕分析指将屏幕作为输入让模型进行处理,屏幕上的图标等进行解析。而文本描述相当于将信息以文本结构的形式输入到模型中。显然后者的泛化能力更差,因为一旦屏幕上的图标等发生了变化,后者必须重新输入,重新为图标设置标签。而前者仍然可以使用视觉匹配完成任务。可以比喻为屏幕分析更像是学会了方法,而文本描述更像是记住了答案,方法可能会出错但是可以迁移到别的题目,答案在特定题目上完全正确但是不能迁移到别的地方。

不足:传统使用supervised fine-tuning(SFT),需要大量高质数据且泛化能力还需要提高

新方法:rule-based reinforcement fine-tuning(RFT),只需要少量数据(相对)就可以训练出不错的效果且泛化能力很强

\section{GUI-R1 Framework}
输入:模型需要完成的高层任务$Q$,当前图像$I$,执行历史记录$H$

输出:候选响应的集合$O = \{o_1, \dots, o_N\}$,其中每个响应包括$o_{act}, o_{text}, o_{point}$,分别对应low level action,输入的文本,需要点击的坐标

对于每个响应,用unified action space reward funciton进行评分得到分数$R = \{r_1, \dots, r_N\}$,然后计算相对优势:$A_i = \frac{r_i - mean\{r_1, \dots, r_N\}}{std\{r_1, \dots, r_N\}}$,其中mean是均值,std是标准差

unified action space:统一动作空间指不同平台(Mobile,Web)可以执行的操作可能不同,但是对它们进行分解成为原子操作,这些原子操作是相同的,从而解决多平台训练的动作空间冲突问题

Format reward:评估模型的输出,使得模型的输出更接近预期的语义和语法格式,本论文中认为输出分为两个板块:<think>和<answer>,则希望输出的范式是:<think>中包含想法,而<answer>中包含可能的执行动作[click, select, enter]以及执行任务需要的input text,input point

Accuracy reward:$R_{acc} = R_{act} + R_{point} + R_{text}$,其中前两者的得分标准是完全一致得$1$分,否则$0$分,后者计算语义分数(语义相似度),若超过$0.5$得$1$,否则为$0$

Response reward:$R_{response} = \alpha R_{acc} + \beta R_{format}$

Data collection and fitering:从各个平台收集数据,然后使用Qwen进行过滤,筛选出1.5K高级数据和140K低级数据,然后抽取其中1.5K数据与高级数据结合成为GUI R1 3K数据集

\section{Expriment}
训练流程:SFT->RFT

对比:Grounding Capability,Low level task,High level task三个方面分别利用对应的评测平台进行 不同方面的评价,对评价分数取均值进行比较

结论:GUI-R1在各个方面表现都很优秀

其他:高质量的数据有助于模型的快速收敛,提高性能,且降低Format reward的系数比有助于改善性能原因在于Format学习较为容易,通常在早期就可以学习的很好

\section{Summary}
GUI-R1可以看作GUI-Agent模型,它负责的主要部分是Agent执行操作的决策部分,给定输入(屏幕截图),根据历史和需要完成的任务输出需要执行的操作(类似于轨迹的输出),然后Agent会执行相应的操作,创新之处在于使用RFT进行训练,用较少的数据集实现更好的性能。

而OS-Genesis则与数据集相关,传统的生成的Trajectory方式有两种,都是Task-Driven,而OS-Genesis使用Interaction-Driven,在使用资源较少的情况下引入评分模型,先探索交互,再用模型组成low level instruction,再用模型组成high level instruction,生成的数据集虽然成效不能完全比肩人类标注,但是也很不错,重点在于减少资源消耗

总的来说前者解决小样本环境下模型性能的高效提升,后者解决低成本下的高质量数据问题

而GUI Agent可以分成两个部分:类似于R1的决策部分,接收任务,根据历史和现有条件给予输出(轨迹),然后是执行模块,根据接收到的轨迹执行对应的操作,这一点在之前的github文档中也有说明,两者共同构成GUI Agent,接收命令执行任务,需要使用trajectory进行训练

补充:在模型训练的RL环节首先先对输出进行评分$R$,由上面的公式分成两部分,乘以对应权重加和得最终分数,然后再使用GRPO最优化进行奖励微调
\end{document}