\documentclass[a4paper,12pt]{article}
\usepackage{ctex}  % 使用中文支持
\usepackage{geometry}
\geometry{a4paper, left=2.5cm, right=2.5cm, top=2.5cm, bottom=2.5cm}
\usepackage{amsmath, amssymb}
\usepackage{enumerate}

\title{OS-Genesis Automating GUI Agent Trajectory Construction via Reverse Task Synthesis}
\author{Yang}

\begin{document}
\maketitle
\section{初读思考}
1. OS-Genesis轨迹生成步骤:
\begin{itemize}
    \item GUI代理感知环境并进行逐步交互
    \item 追溯性探索高质量任务,轨迹探索
    \item 轨迹奖励模型用于确保轨迹的质量
\end{itemize}

2. GUI代理自动执行复杂计算任务所必备的能力,可以通过用trajuctory训练来提高:
\begin{itemize}
    \item 理解用户意图
    \item 规划执行任务
    \item 执行动作
\end{itemize}

3. GUI的trajectory包括:
\begin{itemize}
    \item high level instruction: 类似于用户给出的总指令,定义完成任务的总目标
    \item low level instruction: 经过拆解之后完成的小任务
    \item action: 具体的操作比如点击
    \item state: 分为visual(截屏(screeshot))和textual(用json文件等形式格式化地描述状态信息比如窗口大小,颜色等)
\end{itemize}

4. 传统用于生成trajectory的方法:
\begin{itemize}
    \item human supervision: 需要人工标注high level task和注释,高人力成本且劳动耗费大
    \item model based synthesis: 模型脚本生成,限制多样性,降低质量
\end{itemize}

5. OS-Genesis的创新之处:采用反向任务合成技术(reverse synthesis),首先通过与环境交互探索可以执行什么样的功能,观察到的状态和动作被转化为low level instruction。然后将low level instruction转化为high level instruction从而形成trajectory。然后引入reward model来保证data quality

6. 传统与OS之间的区别总结:
\begin{itemize}
    \item task-driven: 传统的方法,首先制定任务(发个邮件),然后拆解执行,自上向下
    \item interaction-driven: OS,探索可以执行的操作(界面上有file edit save send),然后组合操作形成high level instruction,更像是自下向上,探索更多的未知组合
\end{itemize}

\section{OS-Genesis生成trajectory的具体步骤}
\subsection{interaction-driven functional discovery}
首先进行human-free探索在$\epsilon = web, mobile, etc.$的环境中,从中探索交互功能得到操作集$A = \{CLICK, SCROLL, TYPE\}$

这一步骤和人类实际交互步骤类似,只是收集了潜在的操作可能而无需预定义任务

在这个探索过程中triplet(三元组)$<s_{pre}, a, s_{post}>$会被保存,$s_{pre}$表示执行操作前的状态(比如滑动前屏幕截图),$s_{post}$表示执行操作后的状态

\subsection{reverse task synthesis}
利用annotation model(注释模型)生成low level instruction,具体的操作是将收集到的triplet作为输入输入到某个annotation model(如GPT),根据执行操作前后的状态变化,模型会输出一个instruction。比如当下的操作是向下滑动,那么就可能会生成一个任务:向下滑动菜单展示选项,也就是这个操作可能的应用场景。

将low level instruction转化为high level instruction,利用模型进行操作,结合上下文语意转化为更广泛的任务

最后这些high level instruction在环境中被模型(GPT)执行以获得trajectory并保存起来

\subsection{trajectory reward model}
因为上述提到的反向合成利用了模型,为了避免模型的不可靠性,引入奖励机制。

传统的筛选方法是引入labeler function(标签函数),直接丢弃掉不可靠的trajectory。但是任何trajectory中都包含最基层的交互探索,直接丢弃很浪费,因此引入轨迹特征的评价标准,利用GPT进行评分$R \in [1, 5]$,评价标准是completeness(完整执行任务)和coherence(避免冗余操作,效率)

设计了一个算法,首先得到轨迹的集合$G$,遍历其中的元素$g_i$,取$g_i$的最后三个状态作为$S_{last}$,取所有的low level instruction作为$L_i$,则$R_i = RM(S_{last}, L_i)$得到相应的分数,然后概率化:$P(g_i) = \frac{R_i}{\sum_{j=1}^{n}R_j}$

首先明确模型输入的含义:因为我们想要更高的completeness和coherence,因此我们需要知道它完成了没有,操作是否简洁高效,那么最后三个状态可以看到任务是否完成,而操作的完全集合可以判断简洁性和冗余性,而reward model当下可以作为一个黑盒,可以很好地进行评判。

接下来是分数的意义,分数越高的轨迹说明更好,应该作为优质数据,而分数越低的轨迹说明很差,比如只是在胡乱操作。在训练GUI Agent的时候可以使用RL,把reward score作为reward策略,概率分布越高的采取训练的次数越多

\section{实验操作}
\subsection{背景}
选择Android Control作为评估标准,Android World展示Agent完成任务的可行性,针对的环境有mobile和web,以上两个是针对mobile

\subsection{Baseline Construction And Training}
Zero-shot(零样本训练,即模型还未经过训练直接考试):利用CoT等进行指导提示

Task-Driven:传统的生成轨迹方式

Self-instruction(在没有人为总结的前提下模型对自己的行为进行解释总结,形成任务说明):利用GPT生成

\subsection{Evaluation}
采用两个指标:SR(success rate指完成任务的成功率)和Type(执行操作种类与预期一致的概率),GPT的Zero-shot作为对照组,对其余三个模型分别进行上述三种操作的训练得到轨迹,和OS-Genesis得到的轨迹训练结果进行对比

简单来说变量可以分为两类,一类是模型比如Qwen,InternVL,另一类是trajectory数据来源(Task-Driven,Self-Instruction和OS-Genesis),其中zero-shot作为空白对照,通过用AndroidWorld等工具测试模型执行任务的准确率和冗余性来判断trajectory的质量,进而得出OS-Genesis的优越性

\subsection{Training}
采用两种Training策略:
\begin{itemize}
    \item Planning Training: 输入$s, h_i, c$分别表示多模态输入,high level和历史文本,用$L_1 = -\sum log(p_{\theta}(l|s, h_i, c) * p_{\theta}(a|s, h_i, c))$来提高模型的plan能力
    \item Action Training: $L_2 = -\sum log p_{\theta}(a|l, s, c)$,强化基于$l$的正确操作
\end{itemize}
值得一提的是这两个形式很像熵,因此可以通过最大化这两个指标得到最佳性能

\subsection{Conclusion}
从结果来看,OS-Genesis可以缩小其余模型与GPT之间的差距,用不同的工具进行评判,虽然OS不一直是最佳,但是表现非常突出,最坏也是第二,平均表现水平第一

\section{补充工作}
\subsection{保证多样性}
使用Sentence-BERT算法,计算嵌入指令的平均余弦距离,OS的平均余弦距离最大,因此多样性最高。另一个结论是人为参与能够保证高intruction diversity,但是trajectory diversity很低

\subsection{TRM模型的作用}
引入轨迹奖励模型取代传统的标签过滤,再加上什么都不干形成三种方式进行结果对比,发现TRM在high level上表现出巨大优势,且low level表现也很不错

\subsection{Scale Trajectories}
当trajectory变大的时候模型的SR也在逐渐增大

\subsection{与human annotation之间的差距}
从两个角度比:high level instruction和trajectory,前者OS做得更好,原因是有时候模型会会错意且人类注释可能无法在环境中进行;后者人类更优,但是OS也能做到80\%

\section{总结}
本篇论文讲述了关于OS-Genesis这种新的产生trajectory的方法,与传统的Task-Driven相比效率更高,人力资源消耗更少且效果也很不错。主要分为探索交互可能,利用模型(GPT)总结low level和high level,引入reward model筛选trajectory,两种training策略进行训练,保证了多样性、准确性。
\end{document}